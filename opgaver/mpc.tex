\documentclass{article}
\usepackage[utf8]{inputenc}
\usepackage[a4paper, total={5.5in,8in}]{geometry}
\usepackage{amsmath}
\usepackage{amsthm}
\usepackage{seqsplit}
\usepackage{framed}

\newtheoremstyle{opgavedd}{1em}{1em}{\itshape}{0pt}{\bfseries}{:}{ }{\thmname{#1}\thmnumber{ #2}\thmnote{ (#3)}}

\theoremstyle{opgavedd}
\newtheorem{opgave}{Opgave}[section]

\newcommand\Enc{\ensuremath{\mathsf{Enc}}}
\newcommand\Dec{\ensuremath{\mathsf{Dec}}}
\newcommand\nsmod{\ensuremath{\text{ mod }}}

\begin{document}
\title{Opgavesæt. MasterClass i Matematik}
\author{Aarhus Universitet}
\date{2018}
\maketitle

\section{Introduktion til Sikre Beregninger}
Disse opgaver skal illustrere, hvordan man kan regne på data med maksimal sikkerhed, så at
sige uden at kigge på de tal, man regner på. Her er en intuitiv forklaring på, hvad vi er
ude efter: Lad os sige at en mængde af personer ansat i en virksomhed gerne vil vide, hvad
de tjener i gennemsnit. Ingen af dem er villige til at afsløre for andre, hvor meget de
tjener, men har det OK med at gennesnittet bliver offentliggjort.

\begin{opgave}
  En måde at udregne gennemsnittet er at alle fortælle en særligt betroet person deres
  løn. Denne betroet person udregner gennesnittet og fortælle alle de andre
  resultatet. Er dette en god løsning?
\end{opgave}

I stedet kunne vi gøre følgende: Vi anbringer en lommeregner i en kasse, som er lukket,
men vi sørger for en lille åbning i kassen så man kan stikke en hånd ind og trykke på
tasterne. Men man kan ikke se, hvad der står på displayet. Nu går man efter tur hen til
kassen, stikker hånden ind, taster sin løn, og trykker på plus tasten (vi antager at folk
godt kan fornemme hvor tasterne sidder). Den sidste person trykker på ``$=$''-tasten. Nu
åbner vi kassen og ser, hvad der står på displayet.

Lad os kalde antal personer, der var med for $n$ og summen på displayet for $S$. Det
gennemsnit $G$ vi er ude efter kan alle nu regne ud som $G=S/n$.

Det vi er ude efter her er at holde de private data så private som muligt, i den forstand
at \emph{den eneste nye information, der bliver frigivet, er det resultat vi er ude
  efter.} Den metode, vi bruger for at beregne resultatet, siges at være \emph{sikker},
hvis den opfylder denne betingelse.

\begin{opgave}
  Med denne definition, er det så et problem at det vi ser på lommeregneren, når vi åbner
  kassen, ikke er $G$, men $S$?
\end{opgave}

\section{Sikker Addition med Elektronisk Kommunikation}
Det er ikke voldsomt praktisk at skulle håndtere kasser og lommeregnere, og slet ikke
hvis man ikke kan mødes, men må kommunikere over f.eks.~internettet. Heldigvis findes der
mere praktiske løsninger, og vi beskriver en af dem her. Vi vil se på hvoran man beregner
summen af nogle private tal (det er nok til at finde et gennemsnit, som vi har set).

Vi antager, bare for at gøre det nemmere at regne på, at ingen tjener så meget at summen
$S$ bliver større end 100. Og vi antager at der kun er 3 personer der deltager. Lad os
kalde den $i$'te person der er med for $P_{i}$, og lad hans løn være $x_{i}$. Nu vælger
$P_{i}$ tilfældige tal $a_{i}$, $b_{i}$ modulo $100$. Dvs.~$a_{i}$ og $b_{i}$ er
tilfældigt valgte i intervallet $[0\dots 99]$. Han sætter
$c_{i}=x_{i}-(a_{i}+b_{i})\nsmod{100}$. Dette sikre, at
\[
  x_{i} = (a_{i} + b_{i} + c_{i}) \nsmod 100.
\]
Nu sender han hemmeligt $(a_{i},b_{i})$ til $P_{3}$; $(a_{i},c_{i})$ til $P_{2}$; og
$(b_{i},c_{i})$ til $P_{1}$. ``Hemmeligt'' betyder her at ingen tredje person får
kendskab til $a_{i}$, $b_{i}$ eller $c_{i}$. Det kunne man f.eks.~sikre med RSA
kryptering. Dette kaldes at ``distribuere inputs''. Men bemærk, at da $P_{i}$ selv er en
af $P_{1}, P_{2}, P_{3}$, skal han sende to tal ``til sig selv''---det betyder
selvfølgelig bare at han gemmer dem lokalt på sin egen computer.

\begin{opgave}
  Et tal-eksempel: antag $i=1$, og $x_{1}=23$. Lad os sige at $P_{1}$ har valgt
  $a_{1}=50$, $b_{1}=76$. Beregn $c_{1}$.
\end{opgave}

\begin{opgave}
  Lad os sige folk har følgende data:
  \begin{align*}
    P_{1}: b_{1} = 76,\ c_{1} = \text{??},
    && P_{2} : a_{1} = 50,\ c_{1} = \text{??},
    && P_{3}:a_{1}=50,\ b_{1}=76.
  \end{align*}
  Antag nu at $P_{3}$ forsøger at regne ud hvad $x_{1}$ er. Bemærk at han ikke bare kan
  regne værdien ud direkte: Han har fået $a_{1} = 50$ og $b_{1} = 76$, men han mangler
  $c_{1}$. Nu spekulerer han på om $x_{1}$ kunne være $1$. Hvad skal $c_{1}$ så være?
\end{opgave}

\begin{opgave}
  Samme som forrige opgave, men nu gætter $P_{3}$ på at $x_{1}=47$.
\end{opgave}

\begin{opgave}\label{opg:secure-inputs}
    Argumenter generelt for at $P_{3}$ ingen anelse har om hvad $x_{1}$ er for et
    tal. (Bemærk at med samme argument kan man vise at $P_{2}$ heller ikke har noget som
    helst information om $x_{1}$.)
\end{opgave}

Efter at alle har distribueret deres inputs som beskrevet ovenfor, har hver deltager 6
tal. F.eks.~har $P_{1}$ følgende tal:
\[
  P_{1}:(b_{1},c_{1}),\ (b_{2},c_{2}),\ (b_{3},c_{3}).
\]

\begin{opgave}
  Lad
  \begin{align*}
    A&=(a_{1} + a_{2} + a_{3})\mod{100},\\
    B&=(b_{1} + b_{2} + b_{3})\mod{100},\\
    C&=(c_{1} + c_{2} + c_{3})\mod{100}.
  \end{align*}
  Med de tal $P_{1}$ har modtaget kan han nu beregne $B$ og $C$. Hvilke tal
  kan $P_{2}$ udregne?
\end{opgave}

Det sidste der sker i vores løsning er derfor, at $P_{1}$ beregner $B$ og $C$ tallene til
de andre. Tilsvarende sender $P_{2}$ tallene $A$, $C$ og $P_{3}$ sender $A$, $B$.

\begin{opgave}
  Vis at alle nu kan beregne $S$, som jo var $x_{1} + x_{2} + x_{3}$ ved at vise at
  \[
    S = (A + B + C)\mod 100.
  \]
\end{opgave}

(I kan med fordel nedskrive en samlet oversigt over den løsning vi har set på.)

\clearpage

Spørgsmålet er nu om denne løsning er sikker i samme forstand som den med lommeregneren i
kassen? Vi har allerede set at i den første fase, hvor man distribuerer inputs, er der
ikke blevet afsløret noget: ifølge.~opgave~\ref{opg:secure-inputs} har ingen af deltagerne
nogen som helst information om hvad de andre tjener, selv efter at de har modtaget deres
værdier af $a_{i}$, $b_{i}$, $c_{i}$ fra de andre. Deteneste nye der sker til sidst, er at
$A$, $B$ og $C$ bliver offentliggjort.

\begin{opgave}
  Lad os se på situationen fra $P_{3}$'s synspunkt: han ved allerede hvad $A$, $B$ er. Så
  det eneste nye han får at vide er $C$. Husk at sikkerhed her betyder at det eneste nye
  vi har lov at afsløre er $S$, for det er det resultat vi gerne vil beregne. Vis at hvis
  $P_{3}$ får $S$ udleveret, så kan han nemt beregne $C$.
\end{opgave}

Konklusion: Hvis $P_{3}$ får værdien af $S$ at vide (og det var jo det der var meningen),
så kan han nemt selv regne ud hvad $C$ er, så derfor gør det ikke noget at han får $C$. Vi
kan også konkludere at vi har en metode til at lægge tal sammen på en sikker måde, således
at summen er det eneste der bliver frigivet.

\section{Sikker Multiplikation}
I det forrige har vi kigget på hvordan vi kan lave en metode til at lægge tal sammen på en
sikker måde. Følgende er løsningen vi kom frem til:

\begin{framed}
  \begin{center}
    \textbf{Protokol Sikker Addition}
  \end{center}
  \begin{description}
  \item[Distribuere Inputs.] Hver deltager har som input et tal $x_{i}$. Han vælger
    $a_{i}$, $b_{i}$ og $c_{i}$ tilfældigt så $x_{i} = (a_{i}+b_{i}+c_{i})\nsmod{100}$, og
    sender to af tallene til hver deltager $P_{j}$ (inklusiv sig selv), således at $P_{1}$
    ikke ved noget om $a_{i}$, $P_{2}$ ikke ved noget om $b_{i}$, og $P_{3}$ ikke ved
    noget om $c_{i}$ medmindre $P_{j}$ kender $x_{i}$.

  \item[Lokal Beregning.] Hver deltager $P_{i}$ beregner to af værdierne $A$, $B$, $C$ for
    de to af $A$, $B$ og $C$ som $P_{i}$ kender leddene for.

  \item[Beregn Resultat.] Hver deltager $P_{i}$ offentliggør de to $A$, $B$, $C$-værdier
    han kan beregne, og alle beregner summen $S=(A+B+C)\nsmod{100}$.
  \end{description}
\end{framed}

Det er naturligt at spekulere på om man også kan gange tal sammen på en sikker
måde. Svaret er ja, og vi skal se hvordan. Lad os sige at $P_{1}$ har et tal $x$, og
$P_{2}$ har et tal $y$. Vi vil gerne finde ud af hvad $x\cdot y$ er, uden at nogen får
andet at vide om $x$ og $y$.

Antag at $P_{1}$ distribuerer $x$ på samme måde som vi gjorde tidligere. Det betyder at
han vægler $a_{1}$, $b_{1}$, $c_{1}$ så $x=(a_{1} + b_{1} + c_{1})\nsmod{100}$, og giver
to af tallene til hver af de andre: $P_{1}$ har selv $b_{1}$ og $c_{1}$; $P_{2}$ får
$a_{1}$ og $c_{1}$; og $P_{3}$ får $a_{1}$ og $b_{1}$. Vi antager at $P_{2}$ gør det samme
med $y=a_{2} + b_{2} + c_{2}$. Nu er det klart, at:
\begin{align}
  \nonumber
  x\cdot y &\nsmod{100} \\
  \nonumber
           &= (a_{1} + b_{1} + c_{1})(a_{2} + b_{2} + c_{2}) \nsmod{100}, \\
  \label{eq:giantmess}
                       &= (a_{1}a_{2} + a_{1}b_{2} + a_{1}c_{2} + b_{1}a_{2} + b_{1}b_{2}
                         + b_{1}c_{2} + c_{1}a_{2} + c_{1}b_{2} + c_{1}c_{2}) \nsmod{100}.
\end{align}
Det fantastiske er nu, at når $x$ og $y$ er distribueret som vi har beskrevet, så kan hver
af de summander der indgår ovenfor, beregnes af mindst én af deltagerne.

\begin{opgave}
  Vis at når $x$ og $y$ er distribueret som ovenfor, og ved omskrivning af
  Ligning~\ref{eq:giantmess}, så kan $P_{1}$ beregne et tal $u_{1}$, $P_{2}$ kan beregne
  $u_{2}$ og $P_{3}$ kan beregne $u_{3}$, så
  \[
    x\cdot y \mod{100} = (u_{1} + u_{2} + u_{3}) \mod{100}.
  \]
\end{opgave}

Dette betyder at vi kan finde det resultat vi er ude efter hvis vi på en sikker måde kan
finde summen af $u_{1}$, $u_{2}$ og $u_{3}$, men det er jo netop det vi allerede har
konstrueret!

\section{Dating for Kryptologer}
I dette afsnit møder vi to personer man tit støder på i kryptografi: Alice og Bob. De har
lige mødt hinanden, og Alice overvejer om hun vil på date med Bob, som overvejer noget
lignende med Alice. Men ingen af dem er sikre hvad den anden tænker, så derfor er
problemet at begge er alt for generte til bare at spørge---det ville jo være frygtelig
pinligt hvis hun/han sagde nej!

Heldigvis kan vi med vores metode til sikker multiplikation hjælpe Alice og Bob til at
finde ud af om der er gensidig interesse, uden at det bliver pinligt for nogen: Alice
vælger et tal $x$, sådan at $x=1$ hvis hun er interesseret i Bob, og $x=0$ ellers. På
samme måde vælger Bob et tal $y$ hvor $y=1$ hvis han er interesseret og 0 ellers. Det er
nu nemt at indse at $x\cdot y=1$ gælder præcis hvis begge er interesseret, og ellers er
$x\cdot y=0$.

I sidste afsnit har vi set en sikker metode der kan bruges til at regne $x\cdot y$ ud,
altså en metode der ikke afslører noget som helst om $x$ og $y$, ud over netop $x\cdot
y$. Alice og Bob kan bruge netop sådan en metode til at løse deres problem:

\begin{opgave}
  Er ovenstående løsning sikker, såfremt både Alice og Bob er ærlige? Hvorfor/hvorfor ikke?
\end{opgave}

% \begin{opgave}
%   Vis, at hvis Alice har valgt $x=0$, så ved hun hvad resultatet er, allerede inden vi har
%   udført beregningen. Brug dette til at argumentere for, at hvis vores metode til
%   multiplikation er sikker, og hvis Alice ikke er interesseret, så finder hun aldrig ud af
%   om Bob var interesseret.
% \end{opgave}

Den metode til multiplikation vi har set er for tre deltagere, så det kræver at Alice og
Bob har en tredje person til at hjælpe sig. men bemærk at den tredje person ikke får nogen
som helst information, så vi behøver at stole på at han er diskret.

\newpage

\section{Beskyttelse Mod Snyd}
Indtil nu har vi antaget at spillerne altid følger protokollen. Det er dog ikke altid en
rimelig antagelse, da en spiller nogle gange kan have interesse i at afvige fra
protokollen. I det tilfælde kunne han måske være i stand til at lære noget mere
information end det han bør eller han kunne få beregningen til at give et forkert
resultat.

En måde at beskytte sig imod afvigelser er at indbygge ekstra foranstaltninger i
protokollen. Men først skal vi overveje præcis hvilke afvigelser vi vil og i det hele
taget kan beskytte os imod. Lad os derfor kigge på protokollen for Sikker Addition fra
$P_{1}$'s synsvinkel. $P_{1}$ vælger værdierne $a_{1}$, $b_{1}$ og $c_{1}$ sådan at
$x_{1}=a_{1}+b_{1}+c_{1}\nsmod{100}$, hvorefter han sender $(a_{1},c_{1})$ til $P_{2}$ og
$(a_{1},b_{1})$ til $P_{3}$.

\begin{opgave}
  Betragt følgende to måder hvorpå $P_{1}$ kan snyde:
  \begin{enumerate}
  \item $P_{1}$ vælger i stedet værdier $a_{1}'$, $b_{1}'$, $c_{1}'$ sådan at $x_{1}\neq
    a_{1}'+b_{1}'+c_{1}'\nsmod{100}$.
  \item $P_{1}$ sender $a_{1},c_{1}$ til $P_{2}$ og $a_{1}',b_{1}$ til $P_{3}$ hvor
    $a_{1}\neq a_{1}'$.
  \end{enumerate}
  Hvad kan $P_{1}$'s input være i tilfælde 1? Hvad med tilfælde 2?\footnote{Hint: Tænk på
    vores begreb om sikkerhed som værende den hvor vores protokol skal opføre sig som om
    vi havde en betroet 3.-part, der beregner resultatet ud fra deltagernes input og derefter
    giver de korrekte output til deltagerne.}
\end{opgave}

I de næste opgaver kigger vi på hvordan vi beskytter os imod angreb af type 2.

\begin{opgave}
  Hvordan kan $P_{2}$ beskytte sig mod angreb af type 2?
\end{opgave}

\begin{opgave}
  Argumenter for hvordan vi kan undgå at deltagerne snyder med de værdier de offentliggør
  af $A,B,C$. F.eks., hvordan kan vi sikre os at $P_{1}$ ikke sender en forkert værdi af $B$?
\end{opgave}

\newpage

\section{Opsætning af Sikre Beregninger}

I dette afsnit prøver vi at kombinere det vi har lært om RSA med sikre beregninger. I
input fasen til sikre beregninger er det nødvendigt at sende sine tal til den rette
deltager i fortrolighed. For at give den rette opsætning anbefales det at sætte sig i
grupper af tre med en blok papir hver.

\begin{opgave}
  Beskriv hvad der kan gå galt hvis vi ikke har en PKI sat op som parterne stoler på.
\end{opgave}

\begin{opgave}
  Sid sammen i en gruppe på tre. I er nu hver en spiller i en sikker beregnings
  protokol. Opgaven er at bruge RSA til at sætte sikre beregninger op. Hver spiller gør
  følgende:
  \begin{itemize}
  \item Vælg to primtal større end 10 (men mindre end 100) og kald dem $p$ og $q$. Hold
    disse hemmelige, og beregn $N=pq$.
  \item I har set Eulers phi funktion i dag. Brug jeres viden om denne og største fælles
    divisor ($\mathsf{gcd}$) algoritmen til at finde $ed\equiv1\mod\varphi(N)$.
  \item Offentliggør $(e,N)$ ved at lægge et stykke papir foran jer med disse tal på,
    således det er tydeligt hvad er $e$, og hvad er $N$.
  \item Vælg en hemmelighed i intervallet $[0\dots 99]$ og skriv den ned på en hemmelig
    lap papir.
  \item Vælg to tilfældige tal som når vi distribuerede inputs i de forrige afsnit. Hold
    disse tilfælde hemmelige også.
  \item Krypter nu jeres tilfældige tal og skriv hvem de er til. Vis krypteringen til alle
    midt på bordet. Eksempel: $P_{1}$ offentliggør RSA krypteringer af $a_{1}$ og $c_{1}$
    med $P_{2}$ som modtager. $P_{1}$ offentliggør også krypteringen af $a_{1}$ og $b_{1}$
    til $P_{3}$.
  \end{itemize}
\end{opgave}

\begin{opgave}
  Forsættelse af opgaven ovenfor. Når alle er klar og har dekrypteret deres tilsendte
  værdier kan i prøve at gennemføre additions protokollen ovenfor.
\end{opgave}

\begin{opgave}
  Prøv nu multiplikations protokollen. I kan starter med tallene i oprindeligt
  dekrypterede.
\end{opgave}

\begin{opgave}
  Forsættelse af opgaven. I de forrige to opgaver har i regnet på krypteret tal. Afslør
  for hinanden jeres hemmelige værdier og se om resultatet er rigtigt.
\end{opgave}

\section{Sikre Beregninger med enkelt secret sharing}
Vores løsning hidtil har baseret sig på en bestemt type secret sharing, kaldet
\emph{replikeret secret sharing}, hvor hver deltager modtager to værdier, som om det kun
er én hemmelighed, der er blevet delt. Grunden til det var, at så kunne man nemt både lave
sikker addition og multiplikation. Hvis man f.eks.~kun vil lave addition, kan man faktisk
klare sig med en simplerer løsning:

\begin{opgave}
  Brug princippet om at skrive en hemmelighed som en sum af tilfældige tal til at designe
  en ny protokol til sikker addition hvor hver spiller kun modtager én værdi.
\end{opgave}

\begin{opgave}
  I forrige løsning, hvor mange deltagere skal så gå sammen før de kan bryde protokollen?
\end{opgave}

Ved almindeligt secret sharing bliver det nu meget problematisk at lave
multiplikation. Der skal noget ekstra forarbejde til, som vi vil se på i det
følgende. Først vil vi se på hvordan vi klarer \emph{skalarmultiplikation}. Dvs.~vi vil
gerne gange en delt hemmelighed $x$ med et tal $\alpha$, sådan at vi opnår en secret
sharing af $\alpha\cdot x$.

\begin{opgave}
  Hvis deltagerne har delt $x$ mellem sig og samtidig kender $\alpha$, hvad skal de så
  gøre for at opnå en deling af $\alpha\cdot x$?
\end{opgave}

Lad os nu antage at vi er givet en sikker boks, der på forhånd har delt mellem deltagerne
secret sharings af 3 værdier $a,b,c$ sådan at $a\cdot b = c$. Dvs.~$P_{i}$ har værdierne
$a_{i},b_{i},c_{i}$, hvor $a=a_{1}+a_{2}+a_{3}$, $b=b_{1}+b_{2}+b_{3}$ og
$c=c_{1}+c_{2}+c_{3}$.

\begin{opgave}
  Antag at part $P_{i}$ har værdierne $(x_{i}, y_{i})$. Overbevis jer selv om at der ikke
  bliver afsløret noget information ved at offentliggøre værdierne
  $\varepsilon_{i}=x_{i}-a_{i}$, $\delta_{i}=y_{i}-b_{i}$.
\end{opgave}

\begin{opgave}
  Nu kan vi offentligt beregne
  $\varepsilon=\varepsilon_{1}+\varepsilon_{2}+\varepsilon_{3}$ og
  $\delta=\delta_{1}+\delta_{2}+\delta_{3}$. Skriv nu værdien $x$ ud fra $a$ og
  $\varepsilon$ og skriv $y$ ud fra $b$ og $\delta$. Brug dette til at skrive hvad
  $x\cdot y$ er som et udtryk af $a,b,\varepsilon,\delta$.
\end{opgave}

Dette betyder at vi kan finde resultatet $x\cdot y$ på en sikker måde hvis vi på en sikker
måde kan beregne addition og skalarmultiplkation. Men det er jo det vi allerede har
konstrueret!

\begin{opgave}
  Nedskriv nu en samlet løsning (en protokol) for hvordan man laver en sikker
  multiplikation hvis vi antager at deltagerne i forvejen har delt $a,b$ og $c=a\cdot b$
  imellem sig.
\end{opgave}

\end{document}
%%% Local Variables:
%%% mode: latex
%%% TeX-master: t
%%% End:
