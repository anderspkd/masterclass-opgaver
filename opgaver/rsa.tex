\documentclass{article}
\usepackage[utf8]{inputenc}
\usepackage[a4paper, total={5.5in,8in}]{geometry}
\usepackage{amsmath}
\usepackage{amsthm}
\usepackage{seqsplit}

\newtheoremstyle{opgavedd}{1em}{1em}{\itshape}{0pt}{\bfseries}{:}{ }{\thmname{#1}\thmnumber{ #2}\thmnote{ (#3)}}

\theoremstyle{opgavedd}
\newtheorem{opgave}{Opgave}[section]

\newcommand\Enc{\ensuremath{\mathsf{Enc}}}
\newcommand\Dec{\ensuremath{\mathsf{Dec}}}

\begin{document}
\title{Opgavesæt. MasterClass i Matematik}
\author{Aarhus Universitet}
\date{2018}
\maketitle

\section{Kryptering med RSA}
Her følger en kort opridsning af RSA, som vi senere skal bruge til at lave digitale
signaturer.

\paragraph{Nøgle generering.}
Hvis Bob ønsker at lave et sæt RSA nøgler gøres det som følgende:
\begin{enumerate}

\item Vælg to meget store forskellige (mere end 150 cifre) primtal $p$ og $q$;

\item Beregn $N = pq$ og $\varphi(N)=(p-1)(q-1)$;

\item Vælg et heltal $e$ således at $0 < e < \varphi(N)$ og
  $\mathsf{gcd}(e,\varphi(N))=1$;\footnote{Hvis $\mathsf{gcd}(a,b)=1$, så siges $a$ og $b$
    at være \emph{indbyrdes primske}. Dvs.\@ at $a$ og $b$ ikke har nogen primfaktorer
    tilfælles.}

\item Beregn $d$ således at $ed\equiv 1\mod\varphi(N)$.

\end{enumerate}
Den offentlige nøgle er $pk=(N,e)$ ($pk$ står for ``\emph{public key}'') og den hemmelige
nøgle er $sk=(N,d)$ ($sk$ står for ``\emph{secret key}''). Altså hængelåsen, som alle kan
bruge til at ``låse'' beskeden inde eller kryptere den er $pk$, og den nøgle som Bob kan
låse beskederne op, eller dekryptere dem med, er $sk$.

\paragraph{Kryptering og dekryptering.}
Hvis Alice vil sende en krypteret besked til Bob, og hun kender Bobs offentlige nøgle,
kan hun gøre følgende:

Betragt Alices besked (også kaldet \emph{klartekst}) $m$ som et heltal $< N$. Krypteringen af $m$ skrives som
$\Enc(pk,m)$ og resultatet kaldes en \emph{ciffertekst}, her benævnt med $c$. En
ciffertekst for $m$ med nøgle $pk$ udregnes som:
\[
  c = \Enc(pk, m) = m^{e}\mod N.
\]
Bob kan dekryptere $c$ ved hjælp af $sk$ (skrevet $\Dec(sk,c)$) med følgende
beregning:
\[
  m = \Dec(sk, c) = c^{d}\mod N.
\]

\section{Usikker brug af et sikkert kryptosystem}

Alice har hørt at det er meget svært at bryde RSA kryptosystemet, hvorfor hun vælger at
bruge RSA til at sende en besked til Bob. De gør det på følgende måde:

\begin{itemize}
\item Bob vælger hemmelige primtal $p$ og $q$, og sender $N=pq=836201$ og $e=768641$ til
  Alice;
\item Alfabetets bogstaver nummereres så $\text{A}=0$, $\text{B}=1$, $\dots$,
  $\text{Å}=28$;
\item Hvert enkelt bogstav i beskeden krypteres hvert for sig med Bobs offentlige nøgle
  hvorefter de sendes til Bob;
\item Bob kan med sin hemmelige nøgle dekryptere de modtagne bogstaver, og læse beskeden.
\end{itemize}

\begin{opgave}
  Forestil dig nu at du er Eva, som opsnapper beskeden
  \begin{align*}
    (399644, 407526, 424845, 407526, 391559, 598468&,\\
    138351, 407526, 625324, 72718, 530887, 424845)
  \end{align*}
  Hvis i har en lommeregner, eller computer, kan i nok let faktorisere $N$, som stadig er
  alt for lille, men prøv at finde ud af hvordan man let kan bryde krypteringen, også
  selvom man ikke faktorisere $N$. Hint.\footnote{Hvordan ser beskeden ``E'' ud i krypteret
  tilstand? Hvad med alfabetets andre bogstaver?}
\end{opgave}

En teknik til at modvirke dette problem er, at man holder hemmeligheden i de mindste par
cifre, og så ``fylder op'' med tilfældige tal i de andre cifre indtil man har et tal med
lige så mange cifre som $N$ har før man krypterer. Hvis man kan dekryptere er de
tilfældige tal nemme at fjerne igen. F.eks.~kunne $E=4$ laves om til $4375\mathbf{04}$ før
det krypteres, så kun de to sidste cifre indeholder beskeden.

Når man gør dette krypteres den samme besked (det samme bogstav) ikke altid til det samme
tal, og systemet kan derfor ikke brydes så let.

\section{Beregning af store potenser}
Når man skal regne på de meget store tal der i virkeligheden bruges til RSA, er det
vigtigt at bruge de mest effektive metoder. Selvom computere er hurtige, går det alligevel
alt for langsomt hvis vi ikke tænker os om når vi programmerer dem. Et eksempel: Forestil
jer at i skal beregne $23^{17}\mod 33$ med blyant og papir, eller med lommeregner, der kun
kan de grundlæggende regningsarter. Hvis man går i gang uden at tænke, ville man måske
gøre som antydet her:
\begin{align*}
  23^{17} &\mod 33 = \\
          & 23\cdot 23\cdot 23\cdot 23\cdot 23\cdot 23\cdot 23\cdot 23\cdot
            23\cdot 23\cdot 23\cdot 23\cdot 23\cdot 23\cdot 23\cdot 23\cdot
            23 \mod 33.
\end{align*}
Altså bare gange 23 med sig selv 17 gange. Det virker som en temmelig stor opgave, men vi
har heldigvis en løsning.

\begin{opgave}\label{opg-sqr-n-mul}
  Argumenter for at:
  \begin{align*}
    23^{17}\mod 33 &= (23^{16}\mod 33)\cdot 23 \mod 33, \\
    23^{16}\mod 33 &= (23^{8}\mod 33)(23^{8}\mod 33)\mod 33, \\
    23^{8}\mod 33  &= (23^{4}\mod 33)(23^{4}\mod 33)\mod 33, \\
    & \dots\text{ (fortsæt selv her.)}
  \end{align*}
  Og brug dette til at forklare hvordan $23^{17}\mod 33$ kan beregnes med kun 5
  multiplikationer og divisioner.
\end{opgave}

Opgaven ovenfor er et eksempel på en generel teknik der hedder ``square and multiply''
(opløft til anden, og gang sammen), som bruges i alle computerprogrammer der anvender
RSA. Den er en helt nødvendig forudsætning for at RSA kan bruges til noget i praksis.

For de skarpe knive i skuffen kommer her en opgave der går tættere på denne metode (denne
opgave kan i evt.~vende tilbage til hvis der er tid). Først et faktum: ethvert positivt tal
kan skrives, som en sum af 2-potenser (1,2,4,8, etc.). F.eks.:
\begin{align*}
  1 &= 1; &\quad  2 &= 2; &\quad 3 &= 1 + 2 &\quad 4 &= 4; \\
  5 &= 4 +1; &\quad 6&=2+4; &\quad 7&=2+4+1; &\quad &\dots
\end{align*}
(Dem der kender til binære tal kan måske se hvorfor, men det behøver man ikke at vide
noget om i denne opgave.)

\begin{opgave}
  Brug ovennævnte faktum, og ideen i Opgave \ref{opg-sqr-n-mul} til at designe en metoder
  der beregner $a^{e}\mod N$ for vilkårlige positive tal $a,e$ og $N$. Hvor mange
  multiplikationer og divisioner skal man bruge med jeres metode, hvis $e=1026$? (NB: den
  naive metode bruger 1025 multiplikationer.)
\end{opgave}

\section{Digitale Signaturer med RSA}
Indtil nu har vi kun beskæftiget os med \emph{konfidentialitet}---det vil sige
hemmeligholdelse af beskeder ved hjælp af kryptering. En anden lige så vigtig del af
kryptologi er \emph{autentifikation}---at bevise over for andre, at man er den man påstår
man er, og at bevise at en besked er kommet uændret frem.

Lad os kort overvej hvordan ganske almindelige underskrifter (signaturer) virker, eller i
hvert fald burde virke. Ved at underskrive et dokument:
\begin{itemize}

\item Kan det afgøres, at det er mig personligt der har skrevet under;

\item Er jeg juridisk bundet af indholdet;

\item Kan modetageren, ved fremvisning af dokumentet til en tredjepart, overbevise
  tredjeparten at ovenstående to punkter holder.

\end{itemize}

Først forsøger vi at implementere den digitale version af underskrifter på følgende måde,
der ligner den analoge udgave: Når Alice ønsker at underskrive et dokument, vedhæfter hun
en indscanning af sin underskrift til dokumentet, og sender det til Bob.

\begin{opgave}
  Hvorfor er ovenstående løsningsforlag en meget dårlig ide?
\end{opgave}

Det er meget vigtigt, at underskriften på en eller anden måde hænger sammen med indholdet
af dokumentet og vedkommende der skriver under. Her kommer RSA os endnu en gang til hjælp.

\begin{opgave}
  Argumenter for, at der for RSA gælder:
  \[
    m = \Dec(sk, \Enc(pk, m)) = \Enc(pk, \Dec(sk, m)).
  \]
\end{opgave}

Brug nu dette til at vise hvordan Alice kan sende en kærligheds besked til Bob, sådan at
Bob kan verificere, at den virkelig kommer fra Alice og kan senere bevise dette overfor en
tredje person.

\begin{opgave}
  Alices værste fjende Eva er vild med Bob og ville utrolig gerne erstatte $m$ med en
  meddelelse $m'$ hvor Alice siger farvel for altid til Bob. Argumenter for hvorfor det
  ikke er muligt for Eva.
\end{opgave}

\begin{opgave}
  Lad den offentlige RSA nøgle være $N=33$ og $e=3$, og den hemmelige nøgle $d=7$. Hvad
  er underskriften på meddelelsen $m=2$
\end{opgave}

Ovenfor har stiltiende antaget at Bob allerede har den offentlige nøgle han skal bruge for
at checke Alices underskrift. I praksis er denne nøgle naturligvis nødt til at komme et
eller andet sted fra, f.eks.~kan det være Bob på sin maskine har en database hvor han
kan slå offentlige nøgler op for dem han kommunikerer med. Stort set det samme som
telefonnummerlisten på en mobiltelefon.

\begin{opgave}
  Antag at Bob opbevarer offentlige nøgler i en database, som beskrevet ovenfor på sin
  PC. Alice og Bob er stadig forelskede, og Eva er præcis lige så jaloux som før. Bob har
  offentlige nøgler for både Alice og Eva på sin liste. En nat bryder Eva ind på Bobs PC,
  og får adgang til at se og evt.~manipulere med Bobs liste med offentlige nøgler uden at
  Bob opdager noget. Beskriv hvad hun kan gøre for at ødelægge forholdet mellem Alice og
  Bob. Drag en general konklusion omkring hvordan offentlige signatur nøgler skal
  behandles.
\end{opgave}

% \section{Deling af den hemmelige nøgle}

% Tallene for RSA nøglerne i foregående opgave var meget små. I praksis er tallene der skal
% bruges til RSA så store, at de på ingen måde kan huskes i hovedet.\footnote{F.eks.~så er
%   det $N$, som Google bruger, tallet:
% \seqsplit{2615073280026590252903365549957964178120472343236249548198080246522954541924234663429456876993078700348235101809202951838130346856444181606337830884768225374045320661612723101197638135408804245968278009850296969812268061618736546574489717583278955989272571032083803503141871690563678502151439769436572188296536903222111934432083683774162878318328766219702021097510468911571294522502085862670413368774167596123461771080624398928225252370213091857946607798952266301268832911042058882223}
% }
% Man er derfor nødt til at opbevarer den hemmelige nøgle et eller andet sted, og der er
% derfor en vis risiko for, at den kan blive stjålet.

% Vi skal nu se nærmere på en teknik til at reducere risikoen ved tyveri, ved at dele den
% hemmelige nøgle op i to dele. Det gør man ved at vælge to tal $d_{1}$ og $d_{2}$ helt
% tilfældigt, dog sådan at
% \[
%   d_{1} + d_{2} = d.
% \]
% Nu opbevares disse to tal forskellige steder, f.eks.~på to forskellige computere. I
% eksemplet fra opgave 2.4, hvor $d=7$, kunne vi f.eks.~bruge $d_{1}=17$ og $d_{2}=-10$.

% \begin{opgave}
%   Argumenter for, at hvis nogen får fat i $d_{1}$ eller $d_{2}$, men ikke dem begge, så
%   har vedkommende stadig ingen anelse om hvad $d$ er.
% \end{opgave}

% Ideen er nu at lave en underskrift på $m$ ved, at den computer der har $d_{1}$ beregner
% $m^{d_{1}}\mod N$, og computeren med $d_{2}$ beregner $m^{d_{2}}\mod N$. Begge ``halve
% signaturer'' kan nu samles:

% \begin{opgave}
%   Vis at man kan beregne en signatur på $m$ med den private nøgle $sk=(N,d)$ som:
%   \[
%     (m^{d_{1}} \mod N)(m^{d_{2}} \mod N) \mod N.
%   \]
%   Hint\footnote{Brug at $(a\mod N)(b\mod N)\mod N = a\cdot b \mod N$.}
% \end{opgave}

% \begin{opgave}
%   Brug eksemplet fra før, $d_{1}=17$ og $d_{2}=-10$, til at udregne signaturen på
%   beskeden $m=3$ således at den er gyldig for den tidligere offentlige nøgle med $N=33$
%   og $e=3$. ($e$ kan bruges til at kontrollere at signaturen er korrekt.)
% \end{opgave}



\end{document}

%%% Local Variables:
%%% mode: latex
%%% TeX-master: t
%%% End:
